%-------------------------------------------------------------------------------
%   SECTION TITLE
%-------------------------------------------------------------------------------
\cvsection{Work Experience}

\hypersetup{
    colorlinks=true,
    linkcolor=blue,
    filecolor=magenta,
    urlcolor=awesome,
}



%-------------------------------------------------------------------------------
%   CONTENT
%-------------------------------------------------------------------------------
\begin{cventries}

  \cventry
	{Software Engineer III (l4)}
    {Google}
    {Sunnyvale, CA}
    {Oct 2023 - Present}
    {
      \begin{cvitems}
        \item Reduced damage caused by made-for-abuse accounts by XX\% across Google by challenging potentially abusive users with anti-abuse challenges when trying to access non-essential Google services
		\item Designed and implemented agile challenge policies in Java with configurable arms for a more responsive system to abuse patterns.
		\item Engineered the frontend and backend of the UX flow seen by XXXk users daily, providing explanations for restrictions, enabling users to pass challenges, and ultimately facilitating unrestricted access.
		\item Developed monitoring dashboards, real-time metrics, and SQL queries for impact analysis and issue detection
      \end{cvitems}
    }

  \cventry
	{Software Engineer II (l3)}
    {Google}
    {Sunnyvale, CA}
    {Sep 2022 - Oct 2023}
    {
      \begin{cvitems}
	  \item Worked on the \href{https://support.google.com/accounts/answer/7162782}{Action Protection} team, a subset of Google Sign-In responsible for challenging users attempting sensitive actions like password changes with 2FA challenges to protect against hijacking
	  % \item Overhauled the UX flow shown to non-2FA users to allow them to seamlessly enroll and proceed with the attempted action, unblocking XXk daily
	  \item Streamlined the UX flow for non-2FA users, allowing seamless 2FA enrollment and unblocking XXk users daily.
	  \item Collaborated with clients including Gmail, Payments and Ads to provide design reviews, integration support and land feature requests
      \end{cvitems}
    }

  \cventry
    {Software Engineering Intern}
    {Google}
    {Ireland (Remote)}
    {May 2021 - Aug 2021}
    {
      \begin{cvitems}
      \item Worked on the \href{https://developer.android.com/training/safetynet/attestation}{SafetyNet Attestation API}, an anti-abuse platform written in C++ which assesses device side integrity on Android devices
        \item Designed and implemented a principled way to process device information and produce a new integrity verdict for a new class of device
        % \item Added monitoring metrics and a graphical dashboard to this system to allow the onduty to recognize and respond to scaled abuse
        \item Extended the system that calculates preexisting \href{https://developer.android.com/training/safetynet/attestation\#potential-integrity-verdicts}{integrity verdicts} to make it more configurable, scalable and easier to debug
        \item This new system is on the critical path to assessing over 1 billion devices daily
        % \item During the three month internship, two other team members have already used this new infrastructure to simplify their parts of the system
      \end{cvitems}
    }

  \cventry
    {Software Engineering Intern}
    {Google}
    {Ireland (Remote)}
    {May 2020 - Aug 2020}
    {
      \begin{cvitems}
        \item Created a testing system in C++ for exercising code paths in \href{https://github.com/envoyproxy/envoy}{Envoy}, an open source L7 proxy, with randomized inputs
        % \item Wrote a fuzz target for Envoy's xDS protocol, which provides a centralized infrastructure for distributing config files to Envoy nodes
        \item Implemented an abstract state tracker that maintained the correct state of the system to ensure updates were properly processed
        \item Increased testing coverage over key files by more than 40\% and fixed existing bugs the system uncovered
        % \item Increased testing coverage over key files by more than 40\%
      \end{cvitems}
    }

%---------------------------------------------------------
\end{cventries}
